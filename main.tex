%%%%%%%%%%%%%%%%%%%%%%%%%%%%%%%%%%%%%%%%%
% NIWeek 2014 Poster by T. Reveyrand
% www.microwave.fr
% http://www.microwave.fr/LaTeX.html
% ---------------------------------------
% 
% Original template created by:
% Brian Amberg (baposter@brian-amberg.de)
%
% This template has been downloaded from:
% http://www.LaTeXTemplates.com
%
% License:
% CC BY-NC-SA 3.0 (http://creativecommons.org/licenses/by-nc-sa/3.0/)
%
%%%%%%%%%%%%%%%%%%%%%%%%%%%%%%%%%%%%%%%%%

%----------------------------------------------------------------------------------------
%	PACKAGES AND OTHER DOCUMENT CONFIGURATIONS
%----------------------------------------------------------------------------------------

\documentclass[a0paper,portrait]{baposter}

\usepackage[font=small,labelfont=bf]{caption} % Required for specifying captions to tables and figures
\usepackage{booktabs} % Horizontal rules in tables
\usepackage{relsize} % Used for making text smaller in some places

\usepackage{amsmath,amsfonts,amssymb,amsthm} % Math packages
\usepackage{eqparbox}

\usepackage{textcomp}

\usepackage[brazil]{babel}      % para texto em Portugues
\usepackage[utf8]{inputenc}     % para acentuacao em Portugues com o uso do Unicode,


\graphicspath{{figures/}} % Directory in which figures are stored

 \definecolor{bordercol}{RGB}{140,150,140} % Border color of content boxes
 \definecolor{headercol1}{RGB}{75,0,130} % Background color for the header in the content boxes (left side)
  % indigo: 75,0,130
  % vermelho: 196,45,55
  % purple: 147,112,219
 \definecolor{headercol2}{RGB}{147,112,219} % Background color for the header in the content boxes (right side)
 \definecolor{headerfontcol}{RGB}{255,255,255} % Text color for the header text in the content boxes
 \definecolor{boxcolor}{RGB}{255,255,255} % Background color for the content in the content boxes


\begin{document}

\background{ % Set the background to an image (background.pdf)
\begin{tikzpicture}[remember picture,overlay]
\draw (current page.north west)+(-2em,2em) node[anchor=north west]
{\includegraphics[height=1.1\textheight]{background}};
\end{tikzpicture}
}

\begin{poster}{
grid=false,
borderColor=bordercol, % Border color of content boxes
headerColorOne=headercol1, % Background color for the header in the content boxes (left side)
headerColorTwo=headercol2, % Background color for the header in the content boxes (right side)
headerFontColor=headerfontcol, % Text color for the header text in the content boxes
boxColorOne=boxcolor, % Background color for the content in the content boxes
headershape=roundedright, % Specify the rounded corner in the content box headers
headerfont=\Large\sf\bf, % Font modifiers for the text in the content box headers
textborder=rectangle,
background=user,
headerborder=open, % Change to closed for a line under the content box headers
boxshade=plain
}
{\begin{tabular}{c} \includegraphics[height=1cm]{logomarca.png}\\
\end{tabular}
}  
%
%----------------------------------------------------------------------------------------
%	TITLE AND AUTHOR NAME
%----------------------------------------------------------------------------------------
%
{\bf  \LARGE {Título} \\ % Poster title
\vspace{0.2cm} 
\footnotesize \underline{Autor apresentador}$^1$, Segundo autor$^2$, terceiro autor$^3$ \\  % Author names
\footnotesize $^1$\it {Instituição}\\ $^2$\it{Instituição} \\ $^3$\it{Instituição}\\ % 
\footnotesize $^1$\it{\textcolor{blue}{\underline{email-do-primeiro-autor}}}}
%Author email addresses

{\begin{tabular}{c} \includegraphics[height=0.75cm]{UNIVAP-SF.png}\\
\includegraphics[height=1cm]{ipd2-sf.png}
\end{tabular}
}  

%----------------------------------------------------------------------------------------
%	RESUMO
%----------------------------------------------------------------------------------------
\headerbox{Resumo}{name=resumo,span=1,column=0,row=0}{

O pôster deve ser elaborado para papel a0, em duas ou três colunas, devendo conter, obrigatoriamente o título, o nome dos autores com a sigla das suas respectivas instituições e o email do primeiro autor. 
Será obrigatória a presença do autor apresentador durante a apresentação do pôster.


\textit{Palavras-chave: palavra, palavra, palavra, palavra.}
%Cabe aos autores providenciarem o pôster em material adequado (lona, pvc, glosspaper ou similar) com corda para ser afixado.
}

%---------------------------------------------------------------------------------------
%   INTRODUÇÃO
%---------------------------------------------------------------------------------------

\headerbox{Introdução}{name=introducao,span=1,column=0,below=resumo}{ 
Algumas dicas:
O pôster deverá ter informações referentes à sua pesquisa, informações tais como: Resumo, Introdução, Objetivo, Metodologia, Conclusão e outras informações (estes são pontos de orientação geral e não são regras).
Utilize tamanho de fonte 48 como mínimo para título e fonte 28 como mínimo para conteúdo.
Figuras e tabelas deverão cobrir, no máximo, $50\%$ do pôster, informando a fonte dos dados contidos nas mesmas. A fonte deverá ser colocada abaixo das figuras e tabelas.
As informações apresentadas no pôster devem ser concisas e claras.
Este modelo já se encontra na formatação sugerida.
}


%----------------------------------------------------------------------------------------
%	OBJETIVOS
%----------------------------------------------------------------------------------------



\headerbox{Objetivo}{name=objetivo,span=1,column=0,below=introducao}{ 

Nessa seção deve-se apresentar um parágrafo descrevendo o objetivo geral e alguns itens indicando os objetivos específicos. Pode-se utilizar o ambiente \texttt{itemize} como abaixo.

\begin{itemize}
    \item Objetivo específico 1
    \item Objetivo específico 2
    \item Objetivo específico 3
\end{itemize}

\vspace{9cm}

}

%----------------------------------------------------------------------------------------
%	METODOLOGIA 
%----------------------------------------------------------------------------------------
\headerbox{Metodologia}{name=metodologia,span=1,column=1,row=0}{

Pode-se incluir tabelas utilizando o ambiente \texttt{table} ou conforme apresentado abaixo.
\begin{center}
Tabela: Titulo da Tabela\\
\begin{tabular}{|c|c|}
\hline
    Dado 1 & Dado 2 \\
    Dado 3 & Dado 4  \\
    Dado 5 & Dado 6 \\
    Dado 7 & Dado 8  \\
\hline
\end{tabular}\\
Fonte: do autor.
\end{center}

Enquanto equações são organizadas utilizando o ambiente \texttt{equation}, como segue:

\begin{equation}
    \label{eq:eq1}
    \lim_{x\to 0}\dfrac{\sin x}{x}=1.
\end{equation}

\vspace{0.8cm}

}

%----------------------------------------------------------------------------------------
%	RESULTADO
%----------------------------------------------------------------------------------------
\headerbox{Resultados e Discussões}{name=resultados,span=1,column=1,below=metodologia}{

Pode-se incluir figuras utilizando o ambiente \texttt{includegraphics} ou conforme apresentado abaixo.

\begin{center}
\includegraphics[width=0.8\textwidth]{figures/observatorio.jpg}\\
Figura 1. Posicionar legenda abaixo da imagem, espaçamento simples.\\
Fonte: do autor.
\end{center}

\vspace{1.3cm}
\vspace{9cm}

}

%----------------------------------------------------------------------------------------
%	DISCUSSÃO
%----------------------------------------------------------------------------------------
\headerbox{Conclusão}{name=conclusao,column=2,row=0} % To reduce this block to 1 column width, remove 'span=2'
{Sugestões para fazer uma conclusão:
\begin{itemize}
    \item Fazer um breve resumo do trabalho.
    \item Referir qual foi a grande conclusão do trabalho.
    \item Referir se concretizaram ou não todos os objetivos ou se não foi possível concretizar algum deles e explicar o porquê.
    \item Referir a importância que o trabalho tem para sua pesquisa. 
\end{itemize}

\vspace{1.5cm}

}

%----------------------------------------------------------------------------------------
%	AGRADECIMENTOS
%----------------------------------------------------------------------------------------
\headerbox{Agradecimentos}{name=agradecimentos,column=2,below=conclusao,span=1}{

Agradecimentos a instituições de fomento ou a colaboradores do projeto.
}

%----------------------------------------------------------------------------------------
%	REFERENCIAS
%----------------------------------------------------------------------------------------
\headerbox{Referências}{name=referencias,column=2,below=agradecimentos,span=1}{

Listar as referências listadas no texto. Exemplo de citações:\\

[1] L. Burlaga, \textit{et.al} Magnetic loop behind an interplanetary shock; Voyager, helios, and imp 8 observations. {\it J. Geoph. Res.: Space Physics}, Wiley Online Library, v. 86, n. A8, p. 6673-6684, 1981.

[2] T. Wolfgang, \textit{Introduction: The Ellipsoidal Earth Model.} In: WOLFGANG, Torge. \textit{Geodesy.} 3. ed. New York: De Gruyter, 2001. cap. 1, p. 8.

[3] D. S. Wilks, \textit{Statistical methods in the atmospheric sciences}. Acad. Press San Diego, 1995.

\vspace{9cm}
}

%----------------------------------------------------------------------------------------
%	REALIZAÇÃO
%----------------------------------------------------------------------------------------

\headerbox{Realização}{name=evento,column=0,below=referencias, above=bottom,span=3}{

\begin{center}
    \includegraphics[width=0.7\textwidth]{rodape_simfast.png}
\end{center}


} 


\end{poster}

\end{document}